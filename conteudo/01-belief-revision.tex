%!TeX root=../tese.tex
%("dica" para o editor de texto: este arquivo é parte de um documento maior)
% para saber mais: https://tex.stackexchange.com/q/78101

\chapter{Belief Revision}

\textit{Belief Revision} is a field of Knowledge Representation that aims to understand the epistemic states of an agent. An \textit{epistemic state} of an agent represents all of their beliefs in a given moment \citep{Ribeiro2010}. In other words, Belief Revision seeks to comprehend the agent's behavior when faced with changes in its beliefs \citep{Ribeiro2010,Matos2021}.

The seminal contribution to Belief Revision was made by \citet{AGM1985} who proposed a method for representing the agent's epistemic state alongside a set of operations to manage the changes in its beliefs when receiving new information. The guiding principle behind this method is the \textit{principle of minimal change}, which suggests that a rational agent should adjust its beliefs as little as possible in order to accommodate the new information \citep{Peppas2008}. Due to the authors' names, this model became known as the AGM framework.

\section{Operations}
\label{sec:operations}

In the AGM model, \textit{beliefs} are represented by formulas of a propositional language and epistemic states of an agent are represented by a logically closed sets of propositions, known as \textit{belief sets} \citep{Wassermann2000}. This model defines three operations that can be performed on belief sets: contraction, expansion and revision.

\subsection{Expansion}
\label{subsec:expansion}

An \textit{expansion} operation consists of adding new information to the belief set which might potentially lead us to an inconsistent belief state. If we have a belief set $K$ and a belief $\varphi$, the resulting expansion $K + \varphi$ is obtained by adding the new belief to the initial belief set and computing the logical consequences of the resulting set:

$$K + \varphi = \textsf{Cn}(K \cup \{ \varphi \}),$$

where \textsf{Cn} is a consequence operator and $K \subseteq K + \varphi$, which justifies the term expansion. In essence, an expansion adds a new belief along with all its consequences to the initial belief set. As the operation does not modify any existing beliefs, it may introduce an inconsistency to the entire belief set.

\subsection{Contraction}
\label{subsec:contraction}

A contraction occurs when the agent stops believing in certain information. In simple terms, a \textit{contraction} consists of giving up as many beliefs as it is needed so that the new belief set no longer implies the unwanted belief. Unlike expansion, contractions are characterized by a set of postulates that capture the underlying intuition behind the operation \citep{Wassermann2000}. Let's define a contraction $K - \varphi$, where $K$ is the initial belief set and $\varphi$ is the belief to be removed.

\begin{enumerate}
    \item[--] The result of contracting a belief set is always another belief set:    
    $$K - \varphi \text{ is a belief set \textbf{\textit{(closure)}}}$$

    \item[--] New beliefs would never be added to the belief set during the contraction operation:    
    $$K - \varphi \subseteq K \textbf{\textit{ (inclusion)}}$$

    \item[--] If the belief to be removed is not present in the belief set, no changes occur to the set:
    $$\text{if } \varphi \not \in K, \text{ then } K - \varphi = K \textbf{\textit{ (vacuity)}}$$

    \item[--] On the other hand, unless a belief is a tautology (hence, an element of every belief set), the contracted belief will always be removed:
    $$\text{if } \nvdash \varphi, \textrm{ then } \varphi \notin (K - \varphi) \textbf{\textit{ (success)}}$$
    
    \item[--] A contraction is recoverable by the means of an expansion operation:
    $$K \subseteq (K - \varphi) + \varphi \textbf{\textit{ (recovery)}}$$
    
    \item[--] Contractions made by equivalent beliefs yield equivalent results:
    $$\text{if } \vdash \varphi \leftrightarrow \psi, \text{ then } (K - \varphi) = (K - \psi) \textbf{\textit{ (equivalence)}}$$
\end{enumerate}

Furthermore, besides the basic postulates, the AGM model defines two additional ones:

\begin{enumerate}
    \item [--] A belief $\chi$ that is part of ($K - \varphi$) and ($K - \psi$) will also be part of $K - (\varphi \wedge \psi)$:
    $$(K - \varphi) \cap (K - \psi) \subseteq K-(\varphi \wedge \psi) \textbf{\textit{ (conjunctive overlap)}}$$

    \item [--] Any belief deleted when removing $\varphi$ will also be deleted when removing $\varphi \wedge \psi$:
    $$\text{if } \varphi \notin K - (\varphi \wedge \psi), \text{ then } K - (\varphi \wedge \psi) \subseteq (K - \varphi) \textbf{\textit{ (conjunctive inclusion)}}$$
\end{enumerate}

\subsection{Revision}
\label{subsec:revision}

The process of revision involves adding new information to the belief set while ensuring that the final belief set will be consistent. If a belief set remains consistent after adding a new information, the belief will simply be added into the set. However, if the exapansion results in inconsistency, other parts of the initial belief set may need to be removed to achieve consistency. Similar to contraction, the revision operation $K * \varphi$ is characterized by a set of basic postulates.

\begin{enumerate}
    \item [--] Just like in contraction, the outcome of a revision in a belief set always results in another belief set:
    $$K * \varphi \text{ is a belief set \textbf{\textit{(closure)}}}$$

    \item [--] The sucess of a revision is indicated by a resulting belief set that contains the revised belief:
    $$\varphi \in (K * \varphi) \textbf{\textit{ (success)}}$$
    
    \item [--] No information besides the revised belief and its consequences will be added to the belief set during an operation of revision:
    $$(K * \varphi) \subseteq (K + \varphi) \textbf{\textit{ (inclusion)}}$$
    
    \item [--] If the the new belief is consistent with belief set, all the beliefs are preserverd. At this case, together with inclusion, expansion and revision are equivalents:
    $$\text{if } \neg \varphi \notin K, \text{ then } (K + \varphi) \subseteq (K * \varphi) \textbf{\textit{ (preservation)}}$$
    
    \item [--] The rational agent should aim for consistency at any cost. So, unless the new information in itself is inconsistent, the revised belief set is consistent:
    $$\text{if } \varphi \text{ is consistent then } (K * \varphi) \text{ is also consistent} \textbf{\textit{ (consistency)}}$$
    
    \item [--] Revisions made by equivalent beliefs yield equivalent results:
    $$\text{if } \vdash \varphi \leftrightarrow \psi, \text{ then } (K * \varphi) = (K * \psi) \textbf{\textit{ (equivalence)}}$$
\end{enumerate}

Finally, revision also have two extra postulates to deal with revisions by conjunctions. They say that for any two sentences $\varphi$ and $\psi$, if in revising the initial belief set $K$ by $\varphi$ one is lucky enough to reach a belief set $K * \varphi$ that is consistent with $\psi$, then to produce $K * (\varphi \wedge \psi)$ all that one needs to do is to expand $K * \varphi$ with $\psi$ \citep{Peppas2008}.

\begin{enumerate}
    \item [--] $K * \varphi$ is a minimal change of $K$ to include $\varphi$ and therefore there is no way to arrive at $K * (\varphi \wedge \psi)$ from $K$ with "less change":
    $$K * (\varphi \wedge \psi) \subseteq (K * \varphi) + \psi \textbf{\textit{ (superexpansion)}}$$    
    
    \item [--] If however $\psi$ is consistent with $(K * \varphi)$, these further changes can be limited to simply adding $\psi$ to $(K * \varphi)$ and closing under logical implications:
    $$\text{if } \neg \psi \notin (K * \varphi) \text{ then } (K * \varphi) + \psi \subseteq K * (\varphi \wedge \psi) \textbf{\textit{ (subexpansion)}}$$

\end{enumerate}

\section{Partial meet contraction}

The postulates do not determine a unique contraction or revision operation for a belief set, but rather characterize such operations. \citet{AGM1985} introduces a construction for contraction functions based on the idea of selecting maximal subsets that do not entail the belief being contracted. This operation is known as partial meet contraction.

In \textit{partial meet contraction}, a selection function selects maximal subsets from the belief set that do not entail the sentence being contracted. In other words, the the partial meet contraction is defined as the intersection of all sets selected by the selection function. This construction aims to minimize loss of information from the belief set during contraction. A partial meet contraction satisfies the six basic postulates of contraction in the AGM model.

\theoremstyle{definition}
\newtheorem{definition}{Definition}[chapter]

\begin{definition}
    Let $K$ be a belief set and $\alpha$ be a proposition. \citet{AM1981} defines a \textit{remainder set}, denoted by \textsf{rem}($K$, $\alpha$), as the family of all maximal subsets $B \subseteq K$ that do not imply $\alpha$. In other words, whenever $B \subseteq K$, $B$ is a \textit{remainder} of $K$ with respect to $\alpha$ if and only if both $\alpha \notin$ \textsf{Cn}($B$), and whenever $B' \subseteq K$ and $\textsf{Cn}(B') \cap \alpha = \emptyset$ then $B \not \subseteq B'$.
\end{definition}

\begin{definition}
    Let $K$ be a belief set and let $\gamma$ be any function such that for every proposition $\alpha$, $\gamma(\textsf{rem}(K, \alpha))$ is a non-empty subset of $\textsf{rem}(K, \alpha)$, if the remainder set is non-empty, and $\gamma(\textsf{rem}(K, \alpha)) = \{ K \}$ in the limiting case that the remainder set is empty \citep{AGM1985}. Such a function is a \textit{selection function} for $K$.
\end{definition}

\begin{definition}
    Given the set of propositions $K$ and the selection function $\gamma$, the operation of \textit{partial meet contraction} of the proposition $\alpha$ is defined by putting

    $$K -_{\gamma} \alpha = \bigcap \gamma(\textsf{rem}(K, \alpha))$$

    for all $\alpha$ \citep{AGM1985}.
\end{definition}

\begin{definition}
    The limiting case when for all propositions $\alpha$, the selection function $\gamma$ always pick up only one element of their arguments is called \textit{maxichoice contraction operation} \citep{AGM1985}.
\end{definition}

\section{Belief bases}

The AGM paradigm treated all beliefs in a belief set equally, making no distinction between basic beliefs and those inferred from the basic ones. In \citet{Hansson1993}, \citeauthor{Hansson1993} proposed a significant generalization by introducing belief bases. A \textit{belief base} is a set of propositions that is not closed under logical consequence. Its elements represent beliefs that are held independently of any other belief or set of beliefs \citep{Hansson2022}. In this way, models utilizing belief bases provide enhanced expressiveness compared to the belief set presented in \citet{AGM1985}.

Partial meet contraction was further generalized by \citeauthor{Hansson1993} to apply to belief bases. It is important to note that while partial meet contraction is based on selecting maximal subsets that do not imply the contracted belief, it is not sufficient for these subsets that they do not contain the contracted sentence. \citeauthor{Hansson1993} characterized the partial meet base contraction in the terms of the following postulates:

\begin{enumerate}
    \item [--] Unless the belief is a tautology, the contracted belief will not be part of the set of the consequences derived from the contracted belief base:
    $$\text{if } \nvdash \varphi, \text{ then } \varphi \notin \textsf{Cn}(B - \varphi) \textbf{\textit{ (success)}}$$

    \item [--] New beliefs will never be added to the belief base during the contraction:
    $$B - \varphi \subseteq B \textbf{\textit{ (inclusion)}}$$

    \item [--] Relevance blocks the exclusion of elements that there is no good reason to exclude.
    
    \begin{equation*}
        \begin{split}
            \text{if } \psi \in B \text{ and } \psi \notin  (B - \varphi), \text{ then there is some } B' \text{ such that }  \\
                (B - \varphi) \subseteq B' \subseteq B, \varphi \notin \textsf{Cn}(B') \text{ but } \varphi \in \textsf{Cn}(B' \cup \{ \psi \})  \textbf{\textit{ (relevance)}}
        \end{split}
    \end{equation*}

    \item [--] Uniformity ensures that the result of contracting a sentence from a belief base depends only on which subsets of the belief base imply the contracted sentece. In other words, if all the subsets of $B$ that imply some sentence $\varphi$ also imply $\psi$, and vice versa, then the result of the contractions are equivalents.
    $$\text{if for all subsets } B' \text{ of } B, \varphi \in \textsf{Cn}(B') \text{ if and only if } \psi \in \textsf{Cn}(B'), \text{ then } B - \varphi = B - \psi \textbf{\textit{ (uniformity)}}$$

\end{enumerate}

\newtheorem{theorem}{Theorem}[chapter]

\begin{theorem}
    An operation of contraction of $B$ is a \textit{partial meet base contraction} if and only if it satisfies the four basic postulates above.    
\end{theorem}

\subsection{Kernel contraction}

In \citep{Hansson1994}, \citeauthor{Hansson1994} introduces another construction for obtaining contraction operations, based on the notions of kernels and incision functions. Unlike partial meet contractions that aims for maximal subsets that did not imply the undesired belief, a kernel contraction targets the minimal set of beliefs that do imply the unwanted sentence.

As explained in \citep{Wassermann2000}, the idea behind the kernel contraction is that if we remove at least one element of each minimal subset that implies the belief to be contracted, then we obtain a belief base that does not imply the undesired belief. To perform these removals, we use an incision function.

\begin{definition}
    Let $B$ be a belief base and let $\varphi$ be a proposition. The \textit{kernel set}, denoted by \textsf{ker}($B$, $\varphi$), as the family of all minimal subsets $X \subseteq B$ that entails $\varphi$. In other words, whenever $X \subseteq B$, $X$ is a \textit{kernel} of $B$ with respect to $\varphi$ if and only if both $\varphi \in \textsf{Cn}(X)$ and there is no $Y \subseteq X$ such that $\varphi \in \textsf{Cn}(Y)$.
\end{definition}

\begin{definition}
    Let $B$ be a belief base and let $\sigma$ be any function such that for every proposition $\varphi$, it satisfies the following conditions:
    \begin{itemize}
        \item [--] $\sigma(\textsf{ker}(B, \varphi)) \subseteq \bigcup \textsf{ker}(B, \varphi)$ 
        \item [--] if $X$ is a non-empty element of $\textsf{ker}(B, \varphi)$, then $X \cap \textsf{ker}(B, \varphi) \neq \emptyset$
    \end{itemize}
    
    Such a function is known as an \textit{incision function}, which, in other words, is a function that selects at least one element from each kernel for removal.
\end{definition}

\begin{definition}
    Let $\sigma$ be an incision function. The \textit{kernel contraction} on $B$ determined by $\sigma$ is the operation $-_{\sigma}$ such that for all sentences $\varphi$:

    $$B -_{\sigma} \varphi = B \setminus \sigma(\textsf{ker}(B, \varphi))$$
\end{definition}

\section{Pseudo-contraction}
\subsection{Cn* partial meet pseudo-contraction}
\subsection{Cn* kernel pseudo-contraction}
